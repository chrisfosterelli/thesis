\startchapter{Related Work}
\label{chapter:relatedwork}

\section{Early Semantic Research}

During the reading of a native language, semantic incongruities in sentences 
can be detected based on ERP responses in 
EEG~\cite{kutas1980reading,kuperberg2007neural}. More recently, it has been 
shown that machine learning methods can be used to detect features not clearly 
visible in simple magnitude comparisons. For example, machine learning can be 
used to detect the cognitive state based on functional Magnetic Resonance 
Imaging (fMRI)  data, and can differentiate between trials where participants  
view pictures or read sentences~\cite{Wang2002}. Complex neuronal activation 
features can also be extracted using machine learning for tasks like detecting 
ambiguous sentences, performing sentiment analysis, or determining the category 
of an object~\cite{Mitchell2002,Shinkareva2008,Gu2014}.

\section{Generalizable Semantic Models}

Additionally, machine learning models allow us to more deeply explore semantic 
processing, and have provided impressive insight into the human brain via fMRI 
and Magnetoencephalography (MEG)~\cite{Mitchell2008, Sudre2012}. By training a 
machine learning model to accurately predict the expected fMRI activity of a 
person reading concrete nouns, Mitchell et al. showed that the semantic 
features of a word are correlated with fMRI data of a participant viewing the 
word~\cite{Mitchell2008}. Although the model is trained using observed fMRI 
data of participants reading 60 concrete nouns, the model is capable of 
generating predictions for thousands of words for which it has never seen fMRI 
data. This is achieved by encoding each word as a vector of intermediate 
features based on the co-occurrences of the word with 25 verbs in a  large 
corpus.  This work differs from previous work, which was only capable of 
recognizing brain states that the machine learning algorithm had already been 
trained to 
recognize~\cite{kutas1980reading,kuperberg2007neural,Wang2002,Mitchell2002,Shinkareva2008,Gu2014}. 
Mitchell et al. demonstrated a direct relationship between the statistics of 
word co-occurrence and the neural activation associated with each word's 
meaning~\cite{Mitchell2008}

Another key study in this area reproduced Mitchell et al.~\cite{Mitchell2008} 
using MEG~\cite{Sudre2012}.  Sudre et al. used MEG data and word vectors to 
correctly identify concrete nouns.  However, in this work, the  word vectors 
were based on human responses to semantic questions about the word (e.g. Is it 
alive?  It it bigger than a golf ball?) rather than automatically generated 
features from text corpora. The use of MEG allowed Sudre et al. to pinpoint in 
time when the semantics of a word could be detected and when the strength of 
the representation was the strongest.  Subsequent work showed that, with some 
fine tuning, word vectors derived from a text corpus could be as accurate for 
predicting the word a person is reading as the behavioral vectors used in Sudre 
et al.~\cite{Murphy2012}.

\section{Research in EEG}

EEG data, however, has remained comparatively underutilized for the 
fine-grained distinction of individual words. This may be due to the challenges 
that come with EEG data (e.g. lower spatial resolution, comparatively poor 
signal-to-noise ratio).  One of the first studies to successfully use word 
vectors to differentiate words EEG was performed by Murphy et al in 
2009~\cite{Murphy2009}.  In addition, they were able to distinguish between two 
semantic classes (land mammals or work tools)~\cite{Murphy2009,Murphy2011}. The 
accuracy was as high as 98\% when averaged over multiple analyses, providing 
evidence that EEG could give more cost effective exploration of brain-based 
semantics in more naturalistic environments.  Our study adds to the body of 
evidence that EEG can be used to model semantics representations with 
significant accuracy.

\section{Learning-related Literature}

In addition to studying representation, in this work we also examine learning. 
Our novel experimental design also allows us to study participant learning in a 
unique fashion by applying a machine learning model of semantic 
representations. Learning has been traditionally studied in EEG using ERPs. The 
ERP of particular interest for learning is known as the \emph{reward 
positivity}~\cite{proudfit2015reward}. The amplitude of the reward positivity 
is associated with behavior-measured learning when presented in a reinforcement 
learning paradigm~\cite{holroyd2002neural, sutton1998reinforcement, 
williams2017application}. However, the exact nature of the reward positivity's 
association with learning remains unclear. In some work the reward positivity 
is found to have a progressively reduced amplitude as participants perform 
better on the task, and in other work this correlation has not been 
consistently detected~\cite{walsh2012learning}. We aim to provide an 
alternative tool for analyzing learning in this paradigm, which may provide 
insight into the reward positivity and offer other benefits.

\section{Conclusion}

This paper expands on previous work by adapting the existing semantic analysis 
methodology~\cite{Mitchell2008,Sudre2012} to EEG data.  We use a novel 
experimental design, in which participants perform a reinforcement learning 
task that guides participants through learning an artificial language. To 
detect semantics we trained a machine learning model to map from raw EEG signal 
to word vectors derived from an artificial neural network. We evaluate this 
model using the 2 vs 2 test, a method originally developed by Mitchell et 
al.~\cite{Mitchell2008}, that simplifies a complex multiple regression model 
into a binary classification task. Our results show that 
\begin{inparaenum}[(1)]
  \item we can detect the reward positivity in this learning paradigm
  \item the reward positivity diminishes after the first correct trial
  \item the semantic representation of words is detectable in EEG, and \item we 
  can detect learning as participants progress in through the paradigm,
\end{inparaenum}
closely tracking an ERP analysis.

