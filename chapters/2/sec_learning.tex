\section{Learning-related Literature}

In addition to studying representation, in this work we also examine learning. 
Our novel experimental design also allows us to study participant learning in a 
unique fashion by applying a machine learning model of semantic 
representations. Learning has been traditionally studied in EEG using ERPs. The 
ERP component of particular interest for learning is known as the \emph{reward 
positivity}~\cite{proudfit2015reward}. The reward positivity has also been 
known as the feedback error related negativity (fERN), medial frontal 
negativity (MFN), feedback related negativity (RFN), or feedback negativity 
(FN). This signal is a robust, time-locked ERP component occurring 
approximately 250 ms following error feedback. It is suggested that the reward 
positivity reflects the activity of a generic error monitoring system in the 
brain~\cite{miltner1997event}. It is known to be associated with win/loss 
processing.

The amplitude of the reward positivity is associated with behavior-measured 
learning when presented in a reinforcement learning paradigm such as the one we 
use in this thesis~\cite{holroyd2002neural, sutton1998reinforcement, 
williams2017application}. However, the exact nature of the reward positivity's 
association with learning remains unclear and debated. In some work the reward 
positivity is found to have a progressively reduced amplitude as participants 
perform better on the task, and in other work this correlation has not been 
consistently detected~\cite{walsh2012learning}. We aim to provide an 
alternative tool for analyzing learning in this paradigm, which may provide 
insight into the reward positivity and offer other benefits.
