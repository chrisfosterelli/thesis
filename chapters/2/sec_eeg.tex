\section{Research in EEG}

Compared to fMRI and MEG, EEG data has remained comparatively underutilized for 
the fine-grained distinction of individual words. This may be due to the 
challenges that come with EEG data (e.g. lower spatial resolution, 
comparatively poor signal-to-noise ratio).  One of the first studies to 
successfully use word vectors to differentiate words EEG was performed by 
Murphy et al in 2009~\cite{Murphy2009}. In addition, they were able to 
distinguish between two semantic classes (land mammals or work 
tools)~\cite{Murphy2009,Murphy2011}. The accuracy was as high as 98\% when 
averaged over multiple analyses, providing evidence that EEG could give more 
cost effective exploration of brain-based semantics in more naturalistic 
environments. This thesis adds to the body of evidence that EEG can be used to 
model semantics representations with significant accuracy.
