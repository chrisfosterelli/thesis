\section{Conclusion}

Traditional analysis methods for semantic information in the brain consist 
mostly of ERP-based techniques, however machine learning methods have been able 
to provide additional insight over magnitude based visual comparisons. Mitchell 
et al. built on these early methods to create an approach that utilizes 
semantic word vectors generated from a text corpus~\cite{Mitchell2008}. This 
approach models the actual semantics of words rather than learning a mapping 
between the brain data and a category, and introduces the ability to generalize 
to words the model has never been trained on before. 

This work, original in fMRI, was further iterated on when adapted to 
MEG~\cite{Sudre2012}. It has also been expanded to include more complex 
language structures such as sentences and adjective-noun 
phrases~\cite{Chang2009, pereira2018toward, afyshethesis}. Our work builds on 
these to adapt the corpus-based approach from Mitchell et al. and the 
iterations from Sudre et al. to the EEG collection methodology and our 
reinforcement learning based experiment paradigm. With this new paradigm and 
adapted approach we hope to provide insight into the learning process of the 
brain, something traditionally studied by an ERP component known as the reward 
positivity. The following chapter will describe the experiment paradigm, 
preprocessing techniques, and model framework we use in our experiments.
