\section{Early Semantic Research}

In very early work, semantics in the brain have been analyzed through the 
traditional detection of ERPs in EEG 
data~\cite{kutas1980reading,kuperberg2007neural}. When participants read a 
sentence that involves a semantically inappropriate statement (e.g., he spread 
the warm bread with socks), the brain elicits a measurement ERP response known 
as the N400. This negative component peaks approximately 400 milliseconds after 
stimulus onset, hence the name N400.

While visual inspection of evoked EEG data can be useful for measuring 
phenomena that are directly visible in the data, such as with ERPs, more 
complex patterns can be detected with automated analysis methods such as 
machine learning techniques. This can be useful for identifying attributes of 
the EEG data that are not tied to simple magnitude comparisons, or when 
analysis needs to be performed on an online setting (i.e., in real time).  
Additionally, grand average ERPs can be different in timing and amplitude 
between participants depending on age variations~\cite{cunningham2000speech}.  
For example, an early application of machine learning classification on brain 
data was the use by Wang et al. to detect whether or not participants were 
viewing a picture of reading sentences based on their fMRI 
activity~\cite{Wang2002}. 

Machine learning methods can also be useful for detecting semantic information.  
Mitchell et al. were able to categorize trials of participants reading a word 
into one of twelve semantic categories based on the word in an early 
paper~\cite{Mitchell2002}. Similarly to the research based on ERPs, they could 
also detect when a participant found a sentence to be semantically ambiguous.  
Shinkareva et al. was able to identify individual concepts and their 
corresponding semantic categories for a participant based only on the training 
data from other participants~\cite{Shinkareva2008}. This indicated the 
existence of stable semantic representations of concepts in the brain that are 
shared across people. While much of this previous work was done in fMRI, there 
was similar work using EEG that provided evidence of the ability to identify 
limited semantics. For example, Gu et al. was able to perform sentiment 
analysis (a more simple type of semantic analysis that categorizes concepts 
into positive, negative, or neutral categories) of a limited set of sentences 
using EEG data~\cite{Gu2014}. However, work in this area utilizing EEG did not 
quite match the level of semantic detail that was found in studies using fMRI.
