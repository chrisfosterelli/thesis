\subsection{Data Preprocessing}
  \label{sec:preprocessing}
  All EEG data were processed using Brain Vision Analyzer software (Version 2.1.1, Brain Products GmbH, Munich, Germany). Data from each participant was manually reviewed to identify bad or flat channels due to a poor connection or movement. The channels were marked and removed from the dataset. The data was then downsampled from 500Hz to 250Hz, re-referenced to the average mastoid reference, and put through a dual pass phase free Butterworth filter (pass band: 0.1 Hz to 30 Hz; notch filter: 60 Hz). Epochs were then extracted from the EEG data -1000 ms to 2000 ms around the onset of the symbol. The large time range was to facilitate the correction of eye blinks and movements artifacts via independent component analysis (ICA) provided by Brain Vision Analyzer~\cite{luck2014introduction}. A restricted fast ICA with classic PCA sphering was used to identify components. This process continued until either a convergence bound of 1.0 x 10-7 or 150 steps had been reached. Ocular artifacts were selected manually by inspection of the component head maps and related factor loading, and corrected via ICA back transformation. Electrodes that were initially removed were interpolated via spherical splines. 
  
  Word trials were then re-segmented and trimmed to a 1000ms window following stimulus onset. The data was also baseline corrected using the 200ms prior to stimulus onset. Lastly, artifact rejection was applied. Every exposure that contained an absolute difference between the lowest and highest voltage in that exposure of more than 100ms for any sensor was discarded. Every exposure that contained any period where the increase or decrease on any sensor was more than $10{\mu}V/ms$ was discarded as well.
