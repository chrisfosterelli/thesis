\section{Reward Positivity Experiment}

In order to compare our learning analysis model (described in the previous 
section) with a more traditional ERP based analysis, here we analyze the reward 
positivity in the FCz electrode. We perform a comparison of the correct and 
incorrect responses as well as a comparison of the first six correct responses 
for a word to do this.

\emph{It is important to reiterate here that the Reward Positivity Experiment 
was performed by Chad C. Williams. This section is included in this thesis for 
a contextual comparison with the results found by our new approach.}

\begin{figure}[p]
  \centerline{
    \includegraphics[width=\linewidth]{figures/rewpos}
  }
  \caption[Reward Positivity for Correct and Incorrect Responses]{
    The reward positivity for correct and incorrect responses at the FCz. In 
    both graphics, the y-axis is positive downward. In {\bf A}, we see the 
    signals of the averaged first correct trials for each word and all averaged 
    incorrect trials. In {\bf B}, we see the difference between the averaged 
    first correct trials and all averaged incorrect trials with 95\% confidence 
    intervals. There is a clear presence of the reward positivity. This figure 
    courtesy of Chad C. Williams.
  }
  \label{fig:rewpos}
\end{figure}

\begin{figure}[p]
  \centerline{
    \includegraphics[width=\linewidth]{figures/rewpos_learning}
  }
  \caption[Reward Positivity between the First Six Correct Responses]{
    The reward positivity between the first six correct responses. The y-axis 
    is positive downward for the left subfigure and positive upward for the 
    right subfigure. In {\bf A}, we see the amplitude of the signal at the FCz 
    electrode between the first six averaged correct responses. The amplitude 
    of the correct waveform of the reward positivity is large on the first 
    correct trial and diminishes with subsequent rewards.  The change in this 
    waveform indicates a diminishing reward positivity across learning. In {\bf 
    B} we see the amplitude during the highest period of 228ms - 328ms after 
    stimulus onset compared between the first six averaged correct responses.  
    Again, here we see a clear reward positivity in the first correct trial and 
    a diminishing effect on subsequent correct trials. This figure courtesy of 
    Chad C. Williams.
  }
  \label{fig:rewpos_learning}
\end{figure}

Figure~\ref{fig:rewpos} shows the presence of reward positivity, confirmed by a 
dependent samples t-test of the difference waveform between the first correct 
response and the average of incorrect responses with $p < 0.001$.  
Figure~\ref{fig:rewpos_learning} shows the individual correct trials averaged 
over participants. Here, we see a strong reward positivity for the first 
correct trial and a diminishing effect on subsequent correct trials (fitting a 
power law function with $R^2 = 0.96$). This confirms our hypothesis that the 
reward positivity decreases and the \tvt accuracy increases over trials.
