\section{Participant Performance Experiment}

Here we test if the participants' \tvt accuracies are related to the 
participants' average task accuracies by examining the behavioral data. The 
average task accuracy of individual participants ranges from 72\% - 90\% and 
the mean over all participants is 81\%, and the standard deviation is 4\%.  
While the average task accuracy for participants may not be completely 
comparable across participants, as it may include different number of total 
trials for each participant, it still functions as a representative number of 
the general rate at which the participant learned the mapping. Participants who 
learned the mapping faster would have higher task accuracy in each block and a 
higher average task accuracy, while participants who learned the mapping slower 
would have to repeat more blocks and have their average task accuracy reduced 
by the poor performance on those blocks.

We split the participants into two groups based on their task accuracy, and 
evaluated the \tvt accuracy within these two groups. Because the variance of 
average task accuracy is small across participants, we evaluated small groups 
of top and bottom performers. The average \tvt accuracy in the 0 - 700 ms time 
period of the 7 participants with task accuracy below 80\% is {\bf 59.71\%}, 
and the \tvt accuracy of the 7 top participants (all above 85\% task accuracy) 
was {\bf 65.13\%}. While both of these \tvt accuracies are significantly 
lowered due to the reduction in training data compared to the previous two 
experiments, this suggests a relationship between task performance and our 
ability to detect the semantic meaning of the symbols via EEG. However, this 
effect is less obvious in the 0 - 500 ms time period in which we see respective 
accuracies of 57.47\% and 57.55\%.
