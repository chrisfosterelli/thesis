\section{Participant Learning Experiment}
\label{sec:experiments:participantlearning}

The previous experiment allowed us to detect if semantic representation of 
learned symbols could be detected using EEG. To build on this, we can leverage 
the unique nature of this artificial language paradigm to better understand how 
semantic representations develop as participants learn a language mapping. To 
do this we tested \emph{when} we can detect the semantic mapping, as a function 
of the number of exposures.  Here we determined if we can detect the average 
onset of symbol learning. We compared the \tvt accuracy for the earlier trials 
(e.g. trials 1-3, before the symbol meaning was learned) to the later trials 
(e.g. trials 4-6, after participant learning) to test if we can measure the 
emergence of semantics during the paradigm. As in 
Section~\ref{sec:experiments:semanticrepresentation}, we only considered 
participant-word pairs with six or more exposures, to ensure a fair balance in 
the number of exposures being included in each group. We also cut the trials to 
700 ms and 500 ms as before and followed a similar averaging strategy. We 
compared the \tvt accuracy of averaged overlapping subsets of three exposures, 
selected from the first six exposures.

We hypothesized that the \tvt accuracy would increase in later exposures, as 
the symbol mapping was learned by the participants in the reinforcement 
learning paradigm. It is important to reiterate that we are detecting the 
semantics of the English word, not of a representation for the symbol itself.  
Therefore, accuracy will only increase if participants are able to successfully 
think of the English translation for the symbol. While they were given no 
specific instruction to think of or visualize the English concept, we 
hypothesized they will do this intuitively as a requirement for responding in 
the task.
