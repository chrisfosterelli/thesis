\section{Conclusion}

In this section we introduced the six experiments that we perform on the 
collected dataset. Our first experiment performs a baseline test to see if we 
can detect semantics in EEG using a similar methodology which has been applied 
to fMRI and MEG data. The second experiment expands on this to measure how the 
semantic representations develop over the course of the participants' learning.  
The third experiment performs a traditional reward positivity analysis, so we 
can compare the behaviour between the two analyses. The last two experiments 
break down the analysis in terms of time and sensors, to help us understand 
when the semantic representations peak and from what areas of the brain.

With our collection paradigm described, the experiment framework in place, and 
all of the experiments detailed, the next chapter will discuss the results 
found in these experiments.
