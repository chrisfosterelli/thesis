\section{Participant Performance Experiment}
Reward positivity correlates to participant learning as measured by behavioral 
feedback, so we also validate our measurement using participant responses 
during the paradigm. Recall that we recorded the participants' behavioral 
responses as they learned to map the symbols to English words. Participants 
with higher behavioral accuracy learn the symbol mapping faster, and should 
therefore have a stronger representation of the associated word semantics. As 
in the the Reward Positivity Experiment (Section 
~\ref{sec:experiments:rewpos}), this could provide evidence that we are able to 
detect participant learning, and even quantify the efficacy of learning. We 
hypothesized that the behavioral accuracy of the participants should be 
correlated to the average \tvt accuracy for subjects grouped by behavioral 
accuracy.

To combat the noise inherent in EEG, the \tvt accuracy was calculated over the 
average of several participants. Here we considered two groups: those 7 
participants with the highest and the lowest behavioral accuracies (these 
groups were chosen by the natural grouping in their task accuracies). We cut 
the trials to 700 ms in length and averaged across trials as in the prior two 
experiments. We then calculated \tvt accuracy over these two groups, and 
compared the groups' average task accuracies. We hypothesized the \tvt accuracy 
and behavioral accuracy should be positively correlated.

While this experiment is designed to provide some insight into the relationship 
with task accuracy, segregating the participants into separate groups affects 
the ability of the averaging process to combat noise. This has a progressively 
negative effect on \tvt accuracy, which is important to consider when 
evaluating conclusions from this experiment.
