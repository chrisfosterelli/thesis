\section{Reward Positivity Experiment}
\label{sec:experiments:rewpos}
We also wanted to quantify learning using more traditional learning measurement 
mechanisms. Typically, this measurement is done by comparing the amplitude of 
the reward positivity over trials~\cite{williams2017application}. We expected 
that with this experimental paradigm we would see reward positivity for the 
earlier trials, and diminish thereafter. Our application of \tvt accuracy to 
measure learning is novel.  This more standard analysis is meant to provide 
evidence that participant learning could be detected using the EEG data. We 
hypothesized that both this experiment and the Participant Learning Experiment 
(Section~\ref{sec:experiments:participantlearning}) would show the effects of 
participant learning.

This experiment consisted of two parts. Firstly, we divided the trials into 
groups of correct and incorrect responses then averaged across all participants 
and symbols. We compared the amplitudes of the average signals at the FCz 
electrode, where the reward response is the strongest. Secondly, we compared 
the amplitude of the first six correct responses (averaged in a similar 
fashion) at the same FCz electrode. Note that these six responses \emph{cannot} 
be directly compared to the six responses in the Participant Learning 
Experiment (Section~\ref{sec:experiments:participantlearning}) because the 
Participant Learning Experiment considers all trials, whereas the present 
analysis considers \emph{only correct trials}. To determine 1) the amplitude of 
the reward positivity and 2) the change in correct waveforms as learning 
progresses (first six correct trials), a max peak time was first extracted from 
the reward positivity difference waveform for each participant. An averaged max 
peak at 278ms was found within the 250 ms - 400 ms time range. To extract the 
amplitude of the reward positivity and correct waveforms, we  averaged the data 
+/- 25 ms surrounding this peak.
