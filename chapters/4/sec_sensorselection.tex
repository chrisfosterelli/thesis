\section{Sensor Selection Experiment}

In addition to the timing of semantic representations, we were also interested 
in the \emph{localization} of semantic representations. Thus, we explored the 
\tvt accuracy when only a subset of electrodes were used as input to the 
regression model. 

This is similar to the Time Windowing Experiment, except here we select only 
subset of the electrodes and on which to run the model pipeline. This is an 
additional filtering step on $D_{\text{selected}}$, which reduces the 
dimensions of $D_\text{selected}$ such that $D_\text{selected} \in 
\mathbb{R}^{n_s \times (s_e * l)}$ where the number of selected electrodes is 
defined as $s_e \leq n_e$. 

This process is shown in Figure~\ref{fig:reshape} as the ``Sensor Selection'' 
step. We created $n_s$ groups of sensors, where each sensor has its own group 
that consists of itself and its immediately neighboring sensors. This allowed 
us to explore individual sensor accuracies while being less sensitive to that 
individual sensor's noise. Each sensor has between one and four neighbors under 
our mapping. We can visualize these results using a topographic plot, where the 
value of each electrode is the \tvt accuracy of the corresponding group in 
which that electrode is the main electrode. We ran this analysis for three time 
windows to better understand the localization in separate time periods: 0 - 500 
ms, 500 - 1000 ms, and 0 - 1000 ms.
