\startchapter{Conclusions}
\label{chapter:conclusion}

This chapter brings us to the end of the thesis. Through the previous chapters, 
we have introduced the research area of semantic representation in the human 
brain and covered the key previous work that established the current 
state-of-the-art. We've covered the experiment paradigm in this thesis: a high 
quality dataset of participants performing a reinforcement language learning 
task in which they map symbols to English words and sentences. The model that 
we utilize in this thesis has been covered in detail as well, and the novel 
experiments that we can perform by combining the previous methodolgy used in 
fMRI and MEG, the reinforcement learning experiment paradigm, and the valuable 
features of EEG.

Our contributions confirmed that we can 1) detect the semantic representation 
of English words in EEG while participants read the symbol language, 2) measure 
how these semantic representations develop over time during the participant 
learning phase, 3) validate and compare our model against a traditional reward 
positivity analysis of the same experiment, 4) provide supportive evidence that 
suggests intuitive alignment with the participants' task accuracies, 5) 
identify a delayed peak in the strength of the semantic representation 
correlating to the delay required for the task, and 6) provide further evidence 
that the source of semantic representations in the human brain is highly 
distributed and not simply attributable to a single area of the cortex. This 
research iterates on past research in many ways by providing further support 
for existing evidence, but the largest contributions to the state-of-the-art 
are that EEG is a definite valuable and affordable tool for performing semantic 
representation research even in complex paradigms and that this methodology is 
a powerful approach for analyzing learning in the human brain.

EEG can be used to detect word semantics, even for a symbol-based artificial 
language, and even during the process of learning.  EEG is a more noisy brain 
imaging modality, but even still, we provide evidence that EEG can produce 
decoding results on par with previous work~\cite{Mitchell2008}. We provided 
ERP, behavioral, time, and sensor based evidence that our approach models the 
semantics of the symbol mapping. Our work brings new understanding to the 
dynamics of semantic representations in the brain, and provides evidence that 
we can detect semantic representations \emph{as they are learned}.  This hints 
at several new directions for studying brain function, and the neural 
underpinnings of learning.  In particular, our work is further evidence that 
EEG is an effective tool for studying the brain's mechanisms for semantics and 
learning, and that even fine-grained semantic distinctions can be detected 
using EEG.
