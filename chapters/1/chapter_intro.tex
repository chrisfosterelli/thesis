\startfirstchapter{Introduction}
\label{chapter:introduction}

Each year millions of people will study a foreign language. At first, the 
foreign symbols have no meaning. It is only through dedication and practice 
that these symbols become interpretable concepts. What is happening in the 
brain when we learn a mapping from one language to another? Can we better 
understand this process by measuring waves from the brain? Although the neural 
representations of words have been studied extensively with native languages,
they remain comparatively unexplored during the learning of new languages.
 
Semantics is the branch of linguistics concerned with the study of 
\emph{meaning}. Semanticists study the connection between the symbols, words, 
signs, and phrases we use in language and the conceptual idea that those 
signifiers represent~\cite{kreidler2002introducing}. The study of semantics in 
the brain explores how the brain represents, processes, and learns these 
semantics. These functions are argued to be critical to human cognition and 
communication across all languages and cultures~\cite{croft2004cognitive}.

In this thesis we show that semantic representations of the native word (e.g.,
"book") can be decoded from electrophysiological data measured in the human 
brain using machine learning and a technology known as Electroencephalography 
(EEG).  In the past, this has been done using two other technologies for 
measuring brain activity: functional Magnetic Resonance Imaging (fMRI) and 
Magnetoencephalography (MEG).  Our approach, using EEG, has a number of 
benefits. Particularly, it is often several levels of magnitude cheaper than 
other technologies and generally requires less training to operate.

We also show that, using a similar approach, we can detect the process of 
learning by monitoring how the semantic representations of the native word 
develop during an artificial language learning task. To our knowledge, this is 
the first application of this methodology to research learning in the brain.  
Traditionally, this is done with a technique which measures a specific brain 
response known as the \emph{reward positivity}. Our approach has a number of 
benefits over traditional measurements of learning, including the ability to 
measure the retention of knowledge and the speed at which learning occurs.

\input chapters/1/sec_overview
\input chapters/1/sec_background
\input chapters/1/sec_collection
\input chapters/1/sec_contributions
\input chapters/1/sec_conclusion
