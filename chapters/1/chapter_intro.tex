\startfirstchapter{Introduction}
\label{chapter:introduction}

For most, reading is nearly effortless in our native languages. But, learning a 
new language requires dedication and practice. What is happening in the brain 
as we learn this new mapping of foreign word to familiar concept? Can the 
neural signals underlying these processes be understood by measuring waves from 
the brain? The representations evoked by a native language have been studied 
extensively, but little attention has been paid to the evoked semantic 
representations during the process of learning a new language.
 
Semantics is the branch of linguistics concerned with the study of 
\emph{meaning}. Semanticists study the connection between the symbols, words, 
signs, and phrases we use in language and the conceptual idea that those 
signifiers represent~\cite{kreidler2002introducing}. The study of semantics in 
the brain explores how the brain represents, processes, and learns these 
semantics. These functions are argued to be critical to human cognition and 
communication across all languages and cultures~\cite{croft2004cognitive}.

In this thesis we show that semantic representations of the native word can be 
decoded from electrophysiological data measured in the human brain using 
machine learning and a technology known as Electroencephalography (EEG). In the 
past, this has been done using two other technologies for measuring brain 
activity: functional Magnetic Resonance Imaging (fMRI) and 
Magnetoencephalography (MEG).  Our approach, using EEG, has a number of 
benefits. Particularly, it is often several levels of magnitude cheaper than 
other technologies and generally requires less training to operate.

We also show that, using a similar approach, we can detect the process of 
learning by monitoring how the semantic representations of the native word 
develop during an artificial language learning task. To our knowledge, this is 
the first application of this methodology for learning research in the brain.  
Traditionally, this is done with a technique which measures a specific brain 
response known as the \emph{reward positivity}. Our approach has a number of 
benefits over traditional measurements of learning, including the ability to 
measure the retention of knowledge and the speed at which learning occurs.

\section{Thesis Overview}

The rest of this chapter will introduce the important background concepts 
needed, and detail our specific contributions to the state-of-the-art. The 
chapter following reviews the key literature which we build on in this thesis.  
Chapter 3 introduces the methodologies used in our study, including the data 
collection paradigm, participant information, preprocessing of EEG data, and 
the computational model for analyzing the semantic represenations. With the 
methodologies in mind, Chapter 4 covers the six experiments that we performed 
using our model and Chapter 5 includes the results of those experiments. In 
Chapter 6, we discuss the conclusions that we can draw from our results and how 
the results align with existing literature. The last chapter will summarize our 
new contributions to the field and conclude the thesis.

This research expands on previous work by adapting the existing semantic 
analysis methodology~\cite{Mitchell2008,Sudre2012} to EEG data.  We use a novel 
experimental design, in which participants perform a reinforcement learning 
task that guides participants through learning an artificial language. To 
detect semantics we trained a machine learning model to map from raw EEG signal 
to word vectors derived from an artificial neural network. We evaluate this 
model using the 2 vs 2 test, a method originally developed by Mitchell et 
al.~\cite{Mitchell2008}, that simplifies a complex multiple regression model 
into a binary classification task.

By the end of the thesis, we will have covered results showing that we can 1) 
detect the semantic representation of English words in EEG while participants 
read the symbol language, 2) measure how these semantic representations develop 
over time during the participant learning phase, 3) validate and compare our 
model against a traditional reward positivity analysis of the same experiment, 
4) provide supportive evidence that suggests intuitive alignment with the 
participants' task accuracies, 5) identify a delayed peak in the strength of 
the semantic representation correlating to the delay required for the task, and 
6) provide further evidence that the source of semantic represenations in the 
human brain is highly distributed and not simply attributable to a single area 
of the cortex.

\section{Background and Terminology}

This section will briefly introduce the relevant terminology at a high level, 
for those not familiar with the fields of neuroscience or machine learning.  
With a cursory understanding of these, the topics discussed in the thesis 
should be easier to follow. This is not designed to be a complete review of the 
topics and readers are encouraged to investigate further if a topic is 
unfamiliar to them.

Machine learning is a subfield of artificial intelligence that gives computers 
the ability to improve their performance at a task in response to data about 
that task (called training data). This is done using statistical techniques.  
The most common type of machine learning is supervised learning, in which 
training data consists of pairs of inputs and desired outputs. The algorithm 
learns to map from a given input $x$ to a predicted output $\hat{y}$ using 
training data sets with inputs $X$ and matching outputs $Y$.

In this work we use linear ridge regression, a type of machine learning 
algorithm. Regression indicates that the model predicts a scalar value, rather 
than a category. We use sets of these to make a multi-output prediction, known 
as a multivariate linear regression. A linear model is a model which learns a 
polynomial function with a degree of at most one (that is, it predicts in a 
straight line). Ridge regression, also known as weight decay, is a type of 
linear regression that utilizes a regularization mechanism. This is designed to 
help the algorithm perform better on results which are not in the training set 
(known as generalization). A linear model takes a set of inputs $x_1, x_2, ...  
x_n$ and predicts an output $\hat{y}$ by multiplying weights $w_1, w_2, ...  
w_n$ with the inputs and adding the components together. The weights for a 
linear model can be estimated in a few ways, which are outside of the scope of 
this summary.

Another topic that is discussed are artificial neural networks. These are 
computational systems which are vaguely inspired by the connections in the 
brain. A \emph{neuron} is a single unit which receives inputs $x$ and applies 
weights $w$ to each input respectively (similar to linear models). However, 
after summing the components a special function known as an activation function 
is applied to the result. The result of the activation function is the output 
of the neural. The activation function is typically a non-linear function, 
which allows the artificial neural network to learn to model non-linear data.  
As the phrase "network" suggests, these neurons are generally connected in 
series and parallel to form multiple layers of computation. The weights for a 
neural network are found using a process known as \emph{gradient descent}. 

Some neuroscience terminology is utilized here as well. The \emph{cortex} 
refers to the outermost layer of an organ in the body. In all cases here, we 
are referring to the cerebral cortex, which is the outermost layer of the 
cerebrum. The cerebrum is the upper, largest section of the human brain which 
is associated with higher brain operations such as speech, movement, sensory 
processing, and other functions. Much of this functionality resides directly on 
the cortex. The cortex is categorized into four lobes: the frontal, parietal, 
temporal, and occipital lobes. The cortex contains folds, which increase the 
surface area. Each fold contains a gyrus (the ridge or peak of the fold) and 
sulcus (the depression between one fold and the neighboring fold).

Electrophysiology is the study of electrical activity in biological tissues. In 
neuroscience, the electrophysiology signals of interest are the electrical 
signals from the nervous system of the body.

As mentioned prior, semantics refers to the study of meaning. This research 
area explores how the brain represents these semantics. When we discuss the 
semantic representation of a word in the brain, we are referring to the 
electrophysiological state of the brain while it is processing the meaning of a 
given word.

\section{Collection Methodologies}

In this work we commonly reference three methods of measuring electrophysiology 
activity in different areas of the brain: EEG, fMRI, and MEG. In this section 
we will discuss the measurement function for each, as well as compare the 
benefits and drawbacks between them. EEG is the collection methodology used in 
our work, but a baseline understanding of fMRI and MEG is valuable for 
understanding how our research fits into the state-of-the-art and for comparing 
results between them. 

All three of these methods are referring to \emph{noninvasive} collection, 
which means that they do not require incision into the body to be used.  
Invasive sensors are used for some research, as they can provide a clearer 
signal or capture a smaller selection of neurons than these methods, but are 
generally only used for research on non-human participants. As our research is 
related to the understanding of semantics and language, noninvasive sensors on 
human participants are more ideal.

EEG measures the electrical activity generated by the firing of very large 
groups of neurons in the brain. To do this, electrodes are placed on the scalp 
of the participant with a conductive gel applied. The EEG voltage signal 
represents a difference between the voltages at two electrodes, generally the 
source electrode and an electrode that is identified as the \emph{reference} 
electrode.

MEG measures the magnetic fields generated by the electrical current caused by 
the firing of very large groups of neurons in the brain. To do this, subjects 
put their head into a helmet-shaped opening in the MEG collection device. 
Because the magnetic fields are very sensitive and can be distorted even by the 
earth's natural magnetic field, collection of MEG data must be performed in a 
magnetically shielded room.

fMRI measures the blood-oxygen-level dependent (BOLD) contrast generated by the 
firing of very large groups of neurons in the brain. When neurons fire, they 
require sugar and oxygen to be replenished from the blood stream. This causes a 
change in the magnetism of the blood in the brain tissue, which can be detected 
even deep into the brain at very high levels of detail. This BOLD response is 
measured by generating magnetic fields and measuring the response of the atomic 
nuclei in the blood. However, this effect occurs much slower than detecting the 
direct electrical activity of neurons firing. As this process requires 
generating and measuring magnetic fields, it must be performed in a 
magnetically shielded room and away from metal components.

An overview comparison of the different collection modalities can be found in 
Table~\ref{table:modalities}. Because EEG and MEG both measure at the scalp of 
the participant, they are less capable of measuring subcortical activity in the 
brain than fMRI. Further, the electrical signals measured by EEG do not diffuse 
through the skull and scalp as well as the magnetic fields detected by MEG, 
making EEG more susceptible to noise. However, EEG is sensitive to more brain 
areas than MEG as it detects activity in both the sulci and the top of the 
gyri, while MEG mostly detects activity in the sulci. Due to the lower cost of 
equipment, non-dependence on a magnetically shielded room, and reduced training 
requirements, EEG collection is also more cost effective than fMRI or MEG.  
Despite the challenges with noise, EEG has a lower barrier to research and 
provides a different angle on the activity in the brain, which makes it a 
valuable tool worth exploring for semantic representation research. 

\begin{table}[t]
  \begin{center}
    \def\arraystretch{1.5}
    \begin{tabular}{ |c|c|c|c|c| }
      \hline
      Type & Magnetic Shielding & Spatial Resolution & Temporal 
      Resolution & Cost \\
      \hline
      EEG & Not Required & Low & High & Low \\
      MEG & Required & Medium & High & High \\
      fMRI & Required & High & Low & High \\
      \hline
    \end{tabular}
  \end{center}
  \caption[Collection Methodologies Comparison]{
    A comparison of the different brain data collection methodologies discussed 
    in this thesis.
  }
  \label{table:modalities}
\end{table}

\section{Connected Work}

In addition to the core topics in this thesis, other research projects that are 
related to the topic of semantic representation in the brain were studied and 
not directly included in the main text for this thesis. However, because of 
their connection to the core topic, they have been included as appendices here.

Appendix A describes the work related to the study of the representation of 
music in the human brain. This research similarly explores the application of 
EEG for representation research in the human brain, and utilizes a similar 
approach here. Instead of a language learning task, subjects listen to various 
clips of music while EEG data is recorded. We were able to detect a similar 
relationships between the EEG and audio vectors that we generate as we do in 
this research with EEG and semantic word vectors.

Appendix B describes the work related to the study of the semantic 
representations in convolutional neural networks. Convolutional neural networks 
are a type of artificial neural network which are highly effective at computer 
vision tasks. Because of their inspiration from the structure of neurons in the 
brain, they make an interesting research area for applying these algorithms. 

\section{Contributions}

As with all good science, many people were involved in the making of this 
research. Additionally, much of this research has been published or submitted 
for consideration at a publisher. This thesis will therefore include components 
which have been published or were done by other researchers. This section will 
outline the contributions of everyone and any relevant publications (pending or 
otherwise), so no work is misrepresented.

Chad X, Dr. Olav X, and others from the Neuroeconomics Lab at the University of 
Victoria collected the data for this thesis and performed the EEG preprocessing 
in coordination with us. Chad X performed the reward positivity analysis on the 
data as well. Their EEG expertise was invaluable through many components of the 
project. The word vectors for this thesis were trained by wordvectors.org. The 
work in Appendix A was done in equal collaboration with Dhanush X, Ed X, and 
with menorship from Dr. George X. The work in Appendix B was largely completed 
by Dhanush X, while our contribution was the adversarial component of the 
research. Throughout the entirety of the novel research described in this 
document Dr. Alona Fyshe provided continueing mentoring, guidance, feedback, 
and ideas.

An early version of the work in this thesis was published at the Conference on 
Computational Congnitive Neuroscience. A complete version of the work in this 
thesis has been submitted for consideration at the journal of NeuroImage.  
Appendix A is pending publication at MMSP2018. Appendix B has been submitted 
for consideration at the conference for Neural Information Processing Systems.

\section{Conclusion}

In this chapter we've introduced the high level concepts and research topics 
which will be explored, as well as the technologies and terminologies involved.  
This research shows that semantic representations of the native word can be 
decoded from EEG data when a person views the foreign orthographic form, once 
the participant has successfully learned an artificial language mapping.  We 
use existing methods for detecting semantic representations~\cite{Sudre2012}, 
and apply them in a novel language learning paradigm. We provide supporting 
evidence for this method using event related potential (ERP), behavioral, time, 
and sensor based analysis techniques. In the next chapter, we will detail the 
key background work that we build on in this thesis. 
