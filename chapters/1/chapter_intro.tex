\startfirstchapter{Introduction}
\label{chapter:introduction}

For most, reading is nearly effortless in our native languages. But, learning a 
new language requires dedication and practice. What is happening in the brain 
as we learn this new mapping of foreign word to familiar concept? Can the 
neural signals underlying these processes be understood by measuring waves from 
the brain? The representations evoked by a native language have been studied 
extensively, but little attention has been paid to the evoked semantic 
representations during the process of learning a new language.
 
Semantics is the branch of linguistics concerned with the study of 
\emph{meaning}. Semanticists study the connection between the symbols, words, 
signs, and phrases we use in language and the conceptual idea that those 
signifiers represent~\cite{kreidler2002introducing}. The study of semantics in 
the brain explores how the brain represents, processes, and learns these 
semantics. These functions are argued to be critical to human cognition and 
communication across all languages and cultures~\cite{croft2004cognitive}.

In this thesis we show that semantic representations of the native word can be 
decoded from electrophysiological data measured in the human brain using 
machine learning and a technology known as Electroencephalography (EEG). In the 
past, this has been done using two other technologies for measuring brain 
activity: functional Magnetic Resonance Imaging (fMRI) and 
Magnetoencephalography (MEG).  Our approach, using EEG, has a number of 
benefits. Particularly, it is often several levels of magnitude cheaper than 
other technologies and generally requires less training to operate.

We also show that, using a slightly modified model, we can detect the process 
of learning by monitoring how the semantic representations of the native word 
develop during an artificial language learning task. To our knowledge, this is 
the first application of this methodology for learning research in the brain.  
Traditionally, this is done with a technique which measures a specific brain 
response known as the \emph{reward positivity}. Our approach has a number of 
benefits over traditional measurements of learning, including the ability to 
measure the retention of knowledge and the speed at which learning occurs.

\section{Background and Terminology}

\section{Collection Methodologies}

In this work we commonly reference three methods of measuring electrophysiology 
activity in different areas of the brain: EEG, fMRI, and MEG. In this section 
we will discuss the measurement function for each, as well as compare the 
benefits and drawbacks between them. EEG is the collection methodology used in 
our work, but a baseline understanding of fMRI and MEG is valuable for 
understanding how our research fits into the state-of-the-art and for comparing 
results between them. 

All three of these methods are referring to \emph{noninvasive} collection, 
which means that they do not require incision into the body to be used.  
Invasive sensors are used for some research, as they can provide a clearer 
signal or capture a smaller selection of neurons than these methods, but are 
generally only used for research on non-human participants. As our research is 
related to the understanding of semantics and language, noninvasive sensors on 
human participants are more ideal.

EEG measures the electrical activity generated by the firing of very large 
groups of neurons in the brain. To do this, electrodes are placed on the scalp 
of the participant with a conductive gel applied. The EEG voltage signal 
represents a difference between the voltages at two electrodes, generally the 
source electrode and an electrode that is identified as the \emph{reference} 
electrode.

MEG measures the magnetic fields generated by the electrical current caused by 
the firing of very large groups of neurons in the brain. To do this, subjects 
put their head into a helmet-shaped opening in the MEG collection device. 
Because the magnetic fields are very sensitive and can be distorted even by the 
earth's natural magnetic field, collection of MEG data must be performed in a 
magnetically shielded room.

fMRI measures the blood-oxygen-level dependent (BOLD) contrast generated by the 
firing of very large groups of neurons in the brain. When neurons fire, they 
require sugar and oxygen to be replenished from the blood stream. This causes a 
change in the magnetism of the blood in the brain tissue, which can be detected 
even deep into the brain at very high levels of detail. This BOLD response is 
measured by generating magnetic fields and measuring the response of the atomic 
nuclei in the blood. However, this effect occurs much slower than detecting the 
direct electrical activity of neurons firing. As this process requires 
generating and measuring magnetic fields, it must be performed in a 
magnetically shielded room and away from metal components.


\section{Connected Work}

\section{Contributions}

\section{Conclusion}

In this chapter we've introduced the high level concepts and research topics 
which will be explored, as well as the technologies and terminologies involved.  
This research shows that semantic representations of the native word can be 
decoded from EEG data when a person views the foreign orthographic form, once 
the participant has successfully learned an artificial language mapping.  We 
use existing methods for detecting semantic representations~\cite{Sudre2012}, 
and apply them in a novel language learning paradigm. We provide supporting 
evidence for this method using event related potential (ERP), behavioral, time, 
and sensor based analysis techniques. In the next chapter, we will detail the 
key background work that we build on in this thesis. 
