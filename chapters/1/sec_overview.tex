\section{Thesis Overview}

The rest of this chapter will introduce the important background concepts 
needed, and detail our specific contributions to the state-of-the-art. The 
chapter following reviews the key literature which we build on in this thesis.  
Chapter 3 introduces the methodologies used in this thesis, including the data 
collection paradigm, participant information, preprocessing of EEG data, and 
the computational model for analyzing the semantic representations. With the 
methodologies in mind, Chapter 4 covers the six experiments that we performed 
using our model and Chapter 5 includes the results of those experiments. In 
Chapter 6, we discuss the conclusions that we can draw from our results and how 
the results align with existing literature. The last chapter summarizes our new 
contributions to the field and concludes the thesis.

This research expands on previous work by adapting the existing semantic 
analysis methodology~\cite{Mitchell2008,Sudre2012} to EEG data.  We use a novel 
experimental design, in which participants perform a reinforcement learning 
task that guides them through learning an artificial language. To detect 
semantics we trained a machine learning model to map from raw EEG signals to 
word vectors derived from an artificial neural network. We evaluate this model 
using the \tvt test, a method originally developed by Mitchell et 
al.~\cite{Mitchell2008}, that simplifies a complex multivariate regression 
model into a binary classification task. The \tvt test is done by performing a 
``leave two out'' cross validation in which the two left out vectors are 
predicted and compared to determine if the predicted vectors are closer to 
their own ground truth than they are to the other vector's ground truth. The 
percentage of predictions which are closer to their own ground truth provides a 
measurement of the correlation between the brain data and the word vectors.

In this thesis we show that we can: 1) detect the semantic representation of 
English words in EEG while participants read the symbol language, 2) measure 
how these semantic representations develop over time during the participant 
learning phase, 3) validate and compare our model against a traditional reward 
positivity analysis of the same experiment, 4) provide supportive evidence that 
suggests intuitive alignment with the participants' task accuracies, 5) 
identify a delayed peak in the strength of the semantic representation 
correlating to the delay required for the task, and 6) provide further evidence 
that the source of semantic representations in the human brain is highly 
distributed and not simply attributable to a single area of the cortex.
