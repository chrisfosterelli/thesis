\section{Thesis Overview}

The rest of this chapter will introduce the important background concepts 
needed, and detail our specific contributions to the state-of-the-art. The 
chapter following reviews the key literature which we build on in this thesis.  
Chapter 3 introduces the methodologies used in our study, including the data 
collection paradigm, participant information, preprocessing of EEG data, and 
the computational model for analyzing the semantic represenations. With the 
methodologies in mind, Chapter 4 covers the six experiments that we performed 
using our model and Chapter 5 includes the results of those experiments. In 
Chapter 6, we discuss the conclusions that we can draw from our results and how 
the results align with existing literature. The last chapter will summarize our 
new contributions to the field and conclude the thesis.

This research expands on previous work by adapting the existing semantic 
analysis methodology~\cite{Mitchell2008,Sudre2012} to EEG data.  We use a novel 
experimental design, in which participants perform a reinforcement learning 
task that guides participants through learning an artificial language. To 
detect semantics we trained a machine learning model to map from raw EEG signal 
to word vectors derived from an artificial neural network. We evaluate this 
model using the 2 vs 2 test, a method originally developed by Mitchell et 
al.~\cite{Mitchell2008}, that simplifies a complex multiple regression model 
into a binary classification task.

By the end of the thesis, we will have covered results showing that we can 1) 
detect the semantic representation of English words in EEG while participants 
read the symbol language, 2) measure how these semantic representations develop 
over time during the participant learning phase, 3) validate and compare our 
model against a traditional reward positivity analysis of the same experiment, 
4) provide supportive evidence that suggests intuitive alignment with the 
participants' task accuracies, 5) identify a delayed peak in the strength of 
the semantic representation correlating to the delay required for the task, and 
6) provide further evidence that the source of semantic represenations in the 
human brain is highly distributed and not simply attributable to a single area 
of the cortex.
