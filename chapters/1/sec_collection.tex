\section{Collection Methodologies}
\label{chapter:introduction:sec:collection}

In this work we commonly reference three methods of measuring electrophysiology 
activity in different areas of the brain: EEG, fMRI, and MEG. In this section 
we will discuss the measurement function for each, as well as compare the 
benefits and drawbacks between them. EEG is the collection methodology used in 
our work, but a baseline understanding of fMRI and MEG is valuable for 
understanding how our research fits into the state-of-the-art and for comparing 
the results across collection methodologies.

All three of these methods are referring to \emph{noninvasive} collection, 
which means that they do not require incision into the body to be used.  
Invasive sensors are used for some research, as they can provide a clearer 
signal or capture a smaller selection of neurons than these methods, but are 
generally only used for research on non-human participants. As our research is 
related to the understanding of semantics and language, noninvasive sensors on 
human participants are more common.

EEG measures the electrical activity generated by the firing of very large 
groups of neurons in the brain. To do this, electrodes are placed on the scalp 
of the participant with a conductive gel applied. The EEG voltage signal 
represents a difference between the voltages at two electrodes, generally the 
source electrode and an electrode that is identified as the \emph{reference} 
electrode.

MEG measures the magnetic fields generated by the electrical current caused by 
the firing of very large groups of neurons in the brain. To do this, 
participants put their head into a helmet-shaped opening in the MEG collection 
device.  Collection of MEG data must be performed in a magnetically shielded 
room.

fMRI measures the blood-oxygen-level dependent (BOLD) contrast generated by the 
firing of very large groups of neurons in the brain. When neurons fire, they 
require sugar and oxygen to be replenished from the blood stream. This causes a 
measurable change in the magnetism of the blood. This effect occurs much slower 
than detecting the direct electrical activity of neurons firing, and it also 
must be performed in a magnetically shielded room.

An overview comparison of the different collection modalities can be found in 
Table~\ref{table:modalities}. Because EEG and MEG both measure at the scalp of 
the participant, they are less capable of measuring subcortical activity in the 
brain than fMRI. Further, the electrical signals measured by EEG do not diffuse 
through the skull and scalp as well as the magnetic fields detected by MEG, 
making EEG more susceptible to noise. However, EEG can provide an alternative 
view to MEG as the two modalities respond to spherical sources in the brain 
differently~\cite{cohen1983demonstration}. Due to the lower cost of equipment, 
non-dependence on a magnetically shielded room, and reduced training 
requirements, EEG collection is also more cost effective than fMRI or MEG.  
Despite the challenges with noise, EEG has a lower barrier to research and 
provides a different angle on the activity in the brain, which makes it a 
valuable tool worth exploring for semantic representation research. 

\begin{table}[t]
  \begin{center}
    \def\arraystretch{1.5}
    \begin{tabular}{ |c|c|c|c|c| }
      \hline
      Type & Magnetic Shielding & Spatial Resolution & Temporal 
      Resolution & Cost \\
      \hline
      EEG & Not Required & Low & High & Low \\
      MEG & Required & Medium & High & High \\
      fMRI & Required & High & Low & High \\
      \hline
    \end{tabular}
  \end{center}
  \caption[Collection Methodologies Comparison]{
    A comparison of the different brain data collection methodologies discussed 
    in this thesis.
  }
  \label{table:modalities}
\end{table}
