\section{Comparison with Participant Performance}
In the Participant Performance Experiment, we hypothesized that the task 
accuracy of participants should be related to the \tvt accuracy. To test this, 
we took both the top 7 and bottom 7 performing participants (measured in terms 
of task accuracy) and evaluated the model pipeline with those groups in the 0 - 
700 ms time period.  We found the \tvt accuracy is higher (65.13\% vs 59.71\%) 
for the group that performed better in terms of task accuracy. Higher task 
accuracy implies a participant learned the symbol mapping better than a 
participant with lower task accuracy, and the \tvt results show that this trend 
is measurable in the EEG data.  However, we did not find an effect of similar 
degree for the 0 - 500 ms time period.

However, it should be noted that, because of the set up of our experimental 
pipeline (in particular, averaging over participant EEG data), we cannot 
compute a statistical significance score for the difference in \tvt accuracies 
across the two groups. This also has a further drawback: we likely see the most 
extreme effects when comparing the absolute top and absolute bottom performers, 
but the more data we remove from the groups we compare the lower the \tvt 
accuracy will be. This makes this analysis method less useful here, but it does 
support the hypothesized general trend that the \tvt accuracy should be higher 
for those participants with higher task accuracy.

Further, the experiment paradigm was designed to ensure that participants had 
learned the mapping before progressing through the experiment. Because of this 
design, the average task accuracy of all participants is very close together 
and this type of effect may be difficult to measure. For these reasons we don't 
believe we can confidently confirm the experiment's hypothesis here, and 
additional experimentation is required to more rigorously test and validate 
this theory.
