\section{Comparison with Participant Performance}
In the Participant Performance Experiment, we hypothesized that the task 
accuracy of participants should be related to the \tvt accuracy. To test this, 
we took both the top 7 and bottom 7 performing participants (measured in terms 
of task accuracy) and evaluated the model pipeline with those groups in the 0 - 
700 ms time period.  We found the \tvt accuracy is higher (65.13\% vs 59.71\%) 
for the group that performed better in terms of task accuracy. Higher task 
accuracy implies a participant learned the symbol mapping better than a 
participant with lower task accuracy, and the \tvt results show that this trend 
is measurable in the EEG data.  However, we did not find an effect of similar 
degree for the 0 - 500 ms time period.

The experiment paradigm was designed to ensure that participants had learned 
the mapping before progressing through the experiment. Because of this design, 
the average task accuracy of all participants is very close together and this 
type of effect may be difficult to measure. For these reasons we don't believe 
we can confidently confirm the experiment's hypothesis here, and additional 
experimentation is required to more rigorously test and validate this theory.
