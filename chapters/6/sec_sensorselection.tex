\section{Sensor Selection Analysis}
In the Sensor Selection Experiment, we divided the EEG sensors into groups, and 
evaluated the \tvt accuracy using only the sensors in each group. We visualized 
the \tvt accuracy with a topographic plot, and highlighted the primary 
electrodes for groups that had significant accuracy. The number of electrodes 
in each group differ depending on the number of neighboring electrodes, and 
thus we have varying numbers of features for each \tvt experiment. 
Additionally, the diffusion of EEG signals through the skull and scalp, as well 
as the overlap between neighboring sensors, mean that each region is not 
completely independent in the \tvt test. While it is important to take these 
two factors into consideration when interpreting the accuracies, we will 
explain how the results align with existing literature in other methodologies 
and our other experiments.

We see lower \tvt accuracies for the earlier time period (0--500ms) and higher 
accuracies for the later time period (500ms-1000ms). However, we see the 
highest accuracies when using the whole time window. This suggests that there 
is important semantic information in both windows, in line with the results 
from our Time Windowing Experiment.

We see that the sources for higher accuracies in the full time window are 
distributed across various regions of the cortex. For further insight, we 
compare our results to experiments which use other imaging techniques. Work 
that applies these models in fMRI found the highest scoring voxels were 
distributed across the cortex~\cite{Mitchell2008, pereira2018toward}. Sudre et 
al. found in their MEG-based results that many areas of the cortex contributed 
to semantic representations, but that most were parietal, occipital or 
temporal~\cite{Sudre2012}. When we consider smaller windows in time 
(Figure~\ref{fig:topographic}, A and B) the sensor distribution is more 
left-lateralized and confined to parietal, occipital and temporal regions.  
However, when we consider the full 1000ms, we see significance even in frontal 
electrodes.  Our results align with these studies and collectively provide 
evidence that the semantic representation of words is distributed across the 
cortex. 
