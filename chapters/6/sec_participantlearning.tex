
\section{Measurement of Participant Learning}
We hypothesized that the \tvt accuracy would increase in later exposures for a given word. Our Participant Learning Experiment measured the strength of semantic representation at different exposure counts for individual words, and found that semantic representation of a symbol emerges after or near the 4th trial. The average of trials (4, 5, 6) produced 2 vs 2 accuracy of 72.35\%, which is significantly above chance with $p < 0.001$ (FDR corrected). This is evidence that the symbol mapping is learned very quickly, within only a few exposures. 

Previous work has detected participant learning by grouping trials together and measuring the reward positivity~\cite{krigolson2014we}. However, reward positivity does not always coincide with learning-related behavioral changes (i.e. task accuracy), making it potentially unclear if reward positivity is related to a direct brain function for learning or if reward positivity is an indirect effect from a brain function related to feedback~\cite{walsh2012learning}. Recent work has made new arguments supporting the view that reward positivity is a direct index of neural learning~\cite{williams2017application}. Our work provides another angle for comparison, as we measured learning by detecting the actual \emph{concept} to be learned (i.e. the word vector) rather than using an indirect measurement via an ERP. This supports the claim that reward positivity is directly related to a brain function for learning, or at least indicates evidence for a negative correlation with \tvt accuracy. That is: the reward positivity response decreases in amplitude as the representation for the concepts being learned increases in \tvt accuracy.

Since our model detects the semantic representation, it can be used to model both the process of learning and the later retention of learning. Reward positivity can show when learning has occurred, but the absence of reward positivity does not indicate the knowledge of the topic. Thus, our approach could offer benefits in experiments where it is important to measure \emph{retention} of the mapping. %this is nice! -Alona

A further benefit of our approach is assessment of the speed of learning, in addition to detecting learning processes. Previous literature in learning detected only the presence of learning, but did not aim to quantify the speed at which learning occurs~\cite{krigolson2014we}. Our work shows learning as it occurs over the averaged trials, even in a fairly complicated experimental paradigm. This can be more useful than using task accuracy alone to measure learning, as it is susceptible to guessing.
