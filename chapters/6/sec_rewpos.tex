
\subsection{Comparison with Reward Positivity}
In the Reward Positivity Experiment we compared the amplitude of the reward positivity ERP between both the correct and incorrect responses as well as the amplitude of the first six correct responses. This is not a direct comparison to the Participant Learning Experiment, which includes both incorrect and correct responses in the six trials considered. In this ERP based experiment, we considered only trials where the subject chose correct response.  As anticipated, our results showed a measurable difference in the amplitude of the ERP between the first correct and average incorrect responses. Additionally, we saw a strong response to the first correct feedback and a diminishing response to subsequent correct feedbacks.

These results align with existing studies of the reward positivity in reinforcement learning paradigms, where reward positivity is strong in the first correct trial and then diminishes~\cite{krigolson2014we,krigolson2009learning,bellebaum2014feedback}. They also align with the results seen in our Participant Learning Experiment, as both measurements detect that learning is occurring. Though we cannot directly compare, this provides evidence that our measure of learning (\tvt accuracy) responds similarly to the traditional method of measuring learning through the reward positivity response.
