\section{Time Windowing Analysis}
The Time Windowing Experiment separated the averaged trial data into 50ms 
windows and evaluated the model pipeline on each window. We had hypothesized 
that our time analysis would see a delayed peak response compared to prior 
work~\cite{Sudre2012} due to the additional processing time required to map 
from the symbol language to the English word. We found a peak accuracy in the 
600-650ms window with a 2 vs 2 accuracy of 74.57\%. Previously, a similar 
experiment using MEG data that did not utilize a symbol mapping component found 
a peak accuracy of 350ms-400ms~\cite{Sudre2012}. This delay appears to be 
present in our results, and slows the emergence of semantic representations by 
about 250ms. This provides evidence that the newly formed mapping delays 
retrieval by about a quarter of a second.

Interestingly, very little \tvt accuracy is lost when we evaluated our model 
using only a small window (50ms) compared to our initial experiment (Section 
~\ref{sec:discussion:detectionofsemanticrepresentation}) which used a much 
longer window (700ms). By windowing the exposure we remove the majority of data 
provided to the regression models, but see only a minimal reduction in accuracy 
(5\%). This is notable since a model with significantly fewer parameters is 
much faster to train and evaluate.

We also see a large, statistically significant spike in \tvt accuracy early in 
time, around 125ms-200ms. Sudre et al. had also found statistically significant 
accuracy early in the processing pipeline, but had determined this to be 
heavily correlated to the visual features of the words they presented rather 
than the semantics of the words~\cite{Sudre2012}. Our experiment, however, 
should not experience the same correlation with visual features such as 
\emph{Word Length} or \emph{Image Diagonalness} as we use an arbitrary mapping 
of words to symbols (discussed in 
Section~\ref{sec:discussion:detectionofsemanticrepresentation}). It is notable 
that studies have shown evidence of semantics much earlier in time, such as 
Moseley et al. who were able to differentiate between semantic categories as 
early as 150ms into the exposure~\cite{moseley2013sensorimotor}. There are 
several key difference between this experiment and Sudre et al..  First, our 
experiment uses Skipgram instead of vectors based on behavioral responses to 
semantic questions. The Skipgram method, based on word co-location, encodes 
semantics differently than the behavioral response vectors, and may model some 
component of semantic representations that is available earlier in time.  
Second, our experiment uses multiple parts of speech, whereas the original 
Sudre et al. used only concrete nouns.  Perhaps part of speech identity is 
differentiable early in time, which would not have been detectable in an 
experiment using only nouns.  Third, the task performed by our participants is 
much more interactive and requires engagement learning.  The Sudre et al. task 
was a simple semantic question answering task. Perhaps a more engaging task 
evokes a faster or more consistent neural response.  Each of these differences 
may be the reason we see such early semantic activation.  We will need further 
experiments to determine if this early semantic signal is replicable, and if it 
is truly correlated to very early semantic recall.
