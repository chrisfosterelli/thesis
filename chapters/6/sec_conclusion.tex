\section{Conclusion}

Our experiments have identified some novel results which lead to conclusions 
about both semantic research using EEG and the measurement of learning in the 
brain. As mentioned in the last chapter, we are able to detect semantic 
representation using EEG and measure learning of the artificial language 
mapping using the same methodology. 

The nature of the semantic representation experiment has improvements over 
prior work such as diversifying to symbols rather than words, including a 
larger variety of parts of speech, and the addition of more semantic 
categories. Our approach to measuring learning can offer improvements over the 
traditional reward positivity by measuring the actual concept to be learned 
rather than an ERP correlate. These include the ability to detect retention of 
learning and assess the speed of learning. We also find interesting conclusions 
in our time window analysis and sensor analysis experiments, including the 
identification of an approximately quarter second delay over prior work when 
introducing the translation task and the identification of a distributed nature 
to the semantic representation over the cortex. 

This chapter completes the coverage of our research in detail. The next chapter 
summarizes the thesis and our contributions to the state-of-the-art.
